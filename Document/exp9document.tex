\subsection{实验目的}
理解遥感图像目标识别的概念,理解特征提取与分类的概念,学习主成分分析特征提取方法,以及K近邻分类方法,使用这两种方法完成目标识别。
\subsection{实验原理}
\subsubsection{目标识别的主要步骤}
\begin{figure}[H]
	\centering
	\begin{tikzpicture}[node distance=4cm]
	\node[process](rs_img){目标遥感图像};
	\node[process, right of=rs_img](abstract){特征提取};
	\node[process, right of=abstract](classifiler){分类器};
	\node[process, right of=classifiler](result){分类结果};
	
	\draw[->] (rs_img) -- (abstract);
	\draw[->] (abstract) -- (classifiler);
	\draw[->] (classifiler) -- (result);
	\end{tikzpicture}
\end{figure}
\subsubsection{特征提取}
由原始的观测数据得到变换后的用于分类的特征。变换的主要目的:
\begin{enumerate}
	\item 降低维数
	\item 提高类别可分性
\end{enumerate}
\[ \mathbf{Y}_m=\mathbf{A}_{m\times n}\mathbf{X}_n \]
\subsubsection{主成分分析}
\paragraph{最小重构误差准则}
\begin{description}
	\item[原始特征] $x_i\in \mathbb{R}^d,\quad i=1,2,\dots ,m$,均值为$0$。
	\item[特征变换的目的] 寻找一组新的坐标系$\left\lbrace w_1, w_2,\dots, w_d \right\rbrace$其中每一个向量为标准正交基。
	
	如果丢弃新坐标系中的部分坐标,将维度降低到$d'<d$,则样本在新坐标系中的投影为$\mathbf{z}_i=\left[ z_{i1}, z_{i2},\dots, z_{id} \right]$,重构向量为$\hat{x}_i=\sum_{i=1}^{d}z_{ij}w_j$,我们的目的是:
	$ \min\sum_{i=1}^{m}\left\| \hat{x}_i-x_i \right\|^2 $
	\item[上述问题的解] 样本协方差矩阵的前$d'$个最大的特征值对应的特征向量就是$\left\lbrace w_1, w_2,\dots,w_d \right\rbrace$。
	\item[计算过程]
	\begin{enumerate}
		\item 样本均值变为$0$。
		\item 计算样本协方差矩阵。
		\item 特征值分解,取最大特征值对应的特征向量。
		\item 得到投影矩阵$\mathbf{W}=\left[ w_1, w_2,\dots, w_d \right]$。
	\end{enumerate}
	\item[最大可分性原则] 投影后的向量为:$\mathbf{W}^\mathsf{T}x_i$。使得投影后的向量的方差最大化,推导的结果与前面等价。
	\item[$d'$的选择] $\frac{\sum_{i=1}^{d'}\lambda_i}{\sum_{i=1}^{d}\lambda_i}\leq l$
\end{description}
\paragraph{分类器}
\begin{description}
	\item[K近邻学习] 给定测试样本,基于某种距离度量找出训练样本集中与其最靠近的K个样本,基于K个近邻的信息来进行分类。可使用投票法完成分类,即K个近邻中出现最多的类别作为分类结果。
	 
	$K=1$时,成为最近邻分类器。
\end{description}
\paragraph{分类结果评价}
评价指标:分类准确率。
\subsection{实验流程}
\begin{figure}[H]
	\centering
	\begin{tikzpicture}[node distance=1.5cm]
	\node[startstop](begin){开始};
	\node[io, below of=begin](read_set){读取已标示的数据集};
	\node[process, below of=read_set](cut_img){分割数据集中的图片};
	\node[process, below of=cut_img](split_set){抽取训练样本和测试样本};
	\node[process, below of=split_set](pca){对所有样本进行PCA得到降维矩阵};
	\node[process, below of=pca](feature){对所有样本降维,得到特征向量};
	\node[process, below of=feature](classifying){对测试样本进行K近邻分类};
	\node[process, below of=classifying](accuracy){计算分类准确率};
	\node[io, below of=accuracy](output){输出分类准确率};
	\node[startstop, below of=output](end){结束};
	
	\draw[->] (begin) -- (read_set);
	\draw[->] (read_set) -- (cut_img);
	\draw[->] (cut_img) -- (split_set);
	\draw[->] (split_set) -- (pca);
	\draw[->] (pca) -- (feature);
	\draw[->] (feature) -- (classifying);
	\draw[->] (classifying) -- (accuracy);
	\draw[->] (accuracy) -- (output);
	\draw[->] (output) -- (end);
	\end{tikzpicture}
\end{figure}
\subsection{实验程序}
\lstinputlisting[caption={Images2Mat.m}]{"../Executable Script/Exp 9/Images2Mat2.m"}
\lstinputlisting[caption={CalculatePCAMat.m}]{"../Executable Script/Exp 9/CalculatePCAMat2.m"}
\lstinputlisting[caption={SplitDataSet.m}]{"../Executable Script/Exp 9/SplitDataSet2.m"}
\lstinputlisting[caption={TestClassifierPerformance.m}]{"../Executable Script/Exp 9/TestClassifierPerformence2.m"}
\lstinputlisting[caption={PrincipalComponentAnalysis.m}]{"../Function Library/PrincipalComponentAnalysis.m"}
\subsection{实验结果和分析}
使用上述程序,对原始的$600\times 600\times3$的森林、商业区、海滩和沙漠图片进行分割,成为$60\times 60\times 3$的小图片进行分类测试。

随机抽取不同类别的测试样本各1500个进行分类测试,可以得到如下表所示对分类结果。表中的每一行代表不同类别的测试样本,每一列代表不同类别的测试样本判断为这一分类的概率。所以,表中对角线上的概率即为这一分类的正确率。
\begin{center}
\begin{tabular}{ll|rrrr}
	\toprule
	&&\multicolumn{4}{c}{判定概率} \\
	 & & 森林 & 商业区 & 海滩 & 沙漠 \\
	 \midrule
	 \multirow{4}{*}{源数据类别} & 森林 & 91.93\% & 7.60\% & 0.33\% & 0.13\% \\
	  & 商业区 & 12.00\% & 84.80\% & 2.73\% & 0.47\% \\
	  & 海滩 & 1.07\% & 1.07\% & 96.40\% & 1.47\% \\
	  & 沙漠 & 0.07\% & 1.07\% & 0.13\% & 98.73\% \\
	 \bottomrule
\end{tabular}
\end{center}
由上表不难发现,从宏观上,使用主成分分析法和K近邻分类法对遥感图像进行分类可以取得比较优秀对分类结果,但仔细观察可以发现,对商业区的测试样本的分类还不太理想,对其他分类样本均有令人满意的分类结果。